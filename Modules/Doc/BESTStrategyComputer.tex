\documentclass[a4paper, 11pt]{article}
\begin{document}
\section{Introduction}

This is where I explain how the BEST strategy implementation will work.

\section{Ideas}

\begin{itemize}

\item{Allow for MAD, SAD, NATIVE data collection.}

\item{Decide what to do based on:
\begin{itemize}
\item{Description of the beamline}
\item{Number and species of heavy atoms}
\item{Autoindexing solution}
\item{Spacegroup if available from a reference reflection file}
\item{Sequence.}
\end{itemize}
}

\item{Decide MAD or SAD based on beamline, predicted $f'$ and $f''$, scan,
solvent and number of sites per AA residue.}

\end{itemize}

The input should be a little something like this:

\begin{verbatim}

BEGIN PROJECT TS00

BEGIN CRYSTAL 13185

BEGIN AA_SEQUENCE

MHKMWPSDSNDHRVTRRNVIIFSSLLLGSLAILLALLLIRTKDQYYELRDFALGTSVRIV
VSSQKINPRTIAEAILEDMKRITYKFSFTDERSVVKKINDHPNEWVEVDEETYSLIKAAC
AFAELTDGAFDPTVGRLLELWGFTGNYENLRVPSREEIEEALKHTGYKNVLFDDKNMRVM
VKNGVKIDLGGIAKGYALDRARQIALSFDENATGFVEAGGDVRIIGPKFGKYPWVIGVKD
PRGDDVIDYIYLKSGAVATSGDYERYFVVDGVRYHHILDPSTGYPARGVWSVTIIAEDAT
TADALSTAGFVMAGKDWRKVVLDFPNMGAHLLIVLEGGAIERSETFKLFERE

END AA_SEQUENCE

BEGIN HA_INFO
ATOM SE
NUMBER_PER_MONOMER 5
END HA_INFO

END CRYSTAL 13185


BEGIN CRYSTAL 13186

BEGIN AA_SEQUENCE

MHKMWPSDSNDHRVTRRNVIIFSSLLLGSLAILLALLLIRTKDQYYELRDFALGTSVRIV
VSSQKINPRTIAEAILEDMKRITYKFSFTDERSVVKKINDHPNEWVEVDEETYSLIKAAC
AFAELTDGAFDPTVGRLLELWGFTGNYENLRVPSREEIEEALKHTGYKNVLFDDKNMRVM
VKNGVKIDLGGIAKGYALDRARQIALSFDENATGFVEAGGDVRIIGPKFGKYPWVIGVKD
PRGDDVIDYIYLKSGAVATSGDYERYFVVDGVRYHHILDPSTGYPARGVWSVTIIAEDAT
TADALSTAGFVMAGKDWRKVVLDFPNMGAHLLIVLEGGAIERSETFKLFERE

END AA_SEQUENCE

BEGIN HA_INFO
ATOM SE
NUMBER_PER_MONOMER 5
END HA_INFO

END CRYSTAL 13186

END PROJECT TS00

etc...

\end{verbatim}

This would be interpreted to be two crystals from the same selenomethionated
protein. The first stage would be to collect a couple of reference frames
from each sample and populate a block something like this:

\begin{verbatim}

BEGIN CRYSTAL 13186

...

SCAN 13186_scan.raw

BEGIN SCREENING_IMAGES
WAVELENGTH 0.97950
BEAM 109.0 105.0
IMAGE 13186_test_001.img
DIRECTORY /data/jcsg/als1/8.2.1/20050121/collection/TM1553/13186/
END SCREENING_IMAGES

END CRYSTAL 13186

\end{verbatim}

The screening images should be measured at a distance suitable for the 
resolution you want (inscribed circle) and should be measured at 
$0^{\circ}$, $45^{\circ}$ and $90^{\circ}$. If this is a MAD beamline and
you are interested in a MAD experiment then collect a scan and tell the 
program about this. The scan, together with the results of the screening,
will be used to decide what to measure on each sample.

If a number of samples are available then they should be ranked, then the
weaker samples used to characterise the sample in more detail before collecting
from the best. They should be ranked on the total data collection time for some
``standard'' set, which means that the best crystals are those which will 
give the data in the shortest time, and therefore with the least damage.

If we are doing MAD data collection then measure all wavelengths to the same
resolution (yes this is a waste). If the beamline can change wavelength 
quickly and reliably then the ``best'' way to measure the data is in e.g. 10
degree wedges, else measure data for refinement first away from the edge and
then measure for phasing. Something like LREM -> INFL -> PEAK. 

If SAD then need to decide from the number of sites, the wavelength and 
therefore the anomalous scattering form factors how much data will be needed
to give a successful structure solution. This will also require information
on the solvent fraction and resolution, as increasing these will improve the
chances of density modification producing an interpretable map.

Where to draw these lines will be an interesting area of research. Need to 
probably do some searches of the PDB and also get hold of some SG data.

\section{Magic Equations}

\begin{equation}
\langle \frac{I}{\sigma} \rangle > \frac{1.3 \left( \frac{m}{s} \right) 
^{\frac{1}{2}}}{f''}
\end{equation}

Where $m$ is the mass in Daltons of the ASU and $s$ is the number of sites
in the ASU. $\langle \frac{I}{\sigma} \rangle$ is the mean value over the
whole resolution range - this will need to be accounted for in the SAD strategy
calculation (Holton, Private Communication).

\section{Different Spacegroup Options}

Since the input is relatively straightforward in plain text files, it should be
relatively straightforward to ``tweak'' things to allow calculation of 
strategies for each of the possible pointgroups. Cool.

\end{document}