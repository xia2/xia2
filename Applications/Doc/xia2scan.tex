\documentclass[a4paper, 11pt]{article}
\begin{document}

\section{Introduction}

The application \verb|xia2scan| is designed to scan through a list of 
provided images and print information about the ``quality'' of the 
diffraction on the images. This will end up as an executable
\verb|xia2-scan(=> .sh/.bat)| and a python program xia2scan.py.

\subsection{Appication}

This is designed to be useful for 

\begin{itemize}
\item{Selecting the best images for autoindexing from a sweep.}
\item{Optimizing humidity when a humidifier is available.}
\end{itemize}

\noindent
by printing information about the strength of diffraction in each image.

\section{Uses...}

This uses the wrappers for labelit.screen and labelit.stats\_distl.

\section{Dependencies}

This application will depend on having access to a user provided beam centre
(by implication then an input argument / command line handler) and also 
diffraction image name parsing to allow searching of a directory for
matching images.

\section{Use Cases}

\subsection{UC 1: Humidifier}

Requires:

\begin{itemize}
\item{The correct beam position (for indexing from single images.)}
\item{The images.}
\end{itemize}

Provides:

\begin{itemize}
\item{The statistics for each image in the list.}
\end{itemize}

\subsection{UC 2: Indexing Image Selection}



\end{document}