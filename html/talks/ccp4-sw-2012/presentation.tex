\documentclass[slides,compress]{beamer}
\usepackage{graphicx,amsmath,hyperref}
\usepackage{verbatim}

\usepackage[normalem]{ulem}

\usetheme{default}
\useinnertheme{rectangles}

\title{\huge You can't get the staff - an electronic alternative...}
\subtitle{\large (an introduction to xia2)}

\author{Graeme Winter}
\institute{Diamond Light Source}
\date{CCP4 Study Weekend 2012}

\begin{document}

\setbeamertemplate{background}{
\includegraphics[width=\paperwidth,height=\paperheight]
{diamond-background.png}
}

\frame{\maketitle}

\frame{
\frametitle{Overview}
\begin{itemize}
\item{Background}
\item{What is xia2?}
\item{What does it do and how do I use it?}
\item{What decisions does it make?}
\item{Conclusions / summary}
\end{itemize}
}

\section{Background}

\begin{frame}
\frametitle{Before we start...}
\begin{itemize}
\item{No 
MOSFLM \footnote{A.G.W. Leslie, Acta Cryst. (2006) D62, 48-57},
XDS\footnote{W. Kabsch, Acta Cryst. (2010) D66, 125-132}, 
SCALA\footnote{P. Evans, Acta Cryst. (2006) D62, 72-82}, 
CCP4\footnote{CCP4, Acta Cryst. (1994) D50, 760-763}
}
\item{$\rightarrow$ no xia2}
\item{No 
LABELIT\footnote{N.K. Sauter et al.
J. Appl. Cryst. (2004) 37, 399-409},
CCTBX\footnote{R.W. Grosse-Kunstleve et al.
J. Appl. Cryst. (2002) 35, 126-136},
POINTLESS\footnote{P. Evans, Acta Cryst. (2006) D62, 72-82}, 
etc.}
\item{$\rightarrow$ harder to write xia2, less reliable}
\end{itemize}
\end{frame}

\begin{frame}
\frametitle{Acknowledgements}
\begin{itemize}
\item{Andrew Leslie, Harry Powell, Phil Evans, Wolfgang Kabsch, Kay Diederichs,
Nick Sauter, Ralf Grosse-Kunstleve}
\item{Alun Ashton, Dave Stuart, Diamond beamline staff, Miroslav Papiz, 
Steve Prince, Colin Nave, xia2 users, providers of test data (esp. JCSG)}
\end{itemize}
\end{frame}

\begin{frame}
\frametitle{Background}
\begin{itemize}
\uncover<1->{
\item{Comprehensive, trusted software available}
\item{Background of strong publications (esp. CCP4 study weekends)}
\item{Massive advances in computing}
\item{New synchrotron for UK}
}
\uncover<2->{
\item{$\rightarrow$ a great time to develop automated data reduction}
}
\end{itemize}
\end{frame}

\section{What is xia2?}
\begin{frame}
\frametitle{What is xia2?}
{\large
\uncover<1->{\includegraphics[scale=0.125]{figures/xia2-in-a-nutshell.jpg}}
\uncover<2->{
\includegraphics[scale=0.05]{figures/example-diffraction-image-small.jpg} 
\includegraphics[scale=0.05]{figures/example-diffraction-image-small.jpg}
}
\uncover<3->{$\rightarrow H K L I \sigma_{I}$}
}
\end{frame}

\begin{frame}
\frametitle{What is xia2?}
\begin{itemize}
\uncover<1->{
\item{An \emph{expert} system to perform diffraction data and processing on 
your behalf using your software}
}
\uncover<2->{
\item{A system which can correctly handle multi-pass, multi-wavelength data
sets}
}
\uncover<3->{
\item{\emph{Not} a data processing package}
}
\end{itemize}
\end{frame}

\begin{frame}
\frametitle{Why ``you can't get the staff?''}
\begin{itemize}
\item{12 datasets / hour possible}
\item{Limited help}
\item{Human endurance}
\item{Intended xia2 as tool to delegate data processing to}
\end{itemize}
\end{frame}

\begin{frame}
\frametitle{Why is this useful?}
\begin{itemize}
\item{Second opinion}
\item{Allows you to focus on problem cases}
\item{Help  busy / novice users}
\item{Provides access to other tools}
\item{Reproducible processing}
\end{itemize}
\end{frame}

\begin{frame}
\frametitle{Using xia2}
%{\centerline
\begin{tabular}{c}
{\huge
xia2 -2d /here/are/my/data
}\\
\\
{\huge \emph{- or -}} \\
\\
{\huge
xia2 -3d /here/are/my/data
}\\
\end{tabular}
%}
\end{frame}


\end{document}
